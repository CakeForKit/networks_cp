\chapter{Исследовательская часть}

В данном разделе описано нагрузочное тестирование разработанного веб-сервера.

\section{Технические характеристики}
\begin{itemize}
	\item {Гостевая операционная система --- Ubuntu 22.04.5 LTS (WSL 2)}
	\item {Оперативная память (RAM) -- 8,0 ГБ;}
	\item {Процессор -- AMD Ryzen 7 5800H with Radeon Graphics, 3201 МГц, ядер: 8, логических процессоров: 16;}
\end{itemize}

При проведении замеров времени ноутбук был включен в сеть электропитания, и были запущены только встроенное приложение окружения и система замеров времени.

\section{Цель исследования}
Цель исследования -- определение максимального количества обслуживаемых сетевых соединений и скорости отдачи данных по каждому сетевому соединению и совокупная для разработанного веб-сервера.

\section{Описание исследования}

В ходе исследования производились следующие действия:

\begin{enumerate}[label={\arabic*)}]
	\item Поднимался разработанный веб-сервер на локальном хосте;
	\item Проводилось нагрузочное тестирование с помощью инструмента wrk с заданными параметрами: количество параллельных соединений, время работы;
	\item По завершении тестирования веб-сервер останавливался;
	\item Результаты тестирования сохранялись;
	\item Шаги 1–4 повторялись 10 раз для компенсации случайных отклонений;
	\item Для каждого набора параметров вычислялось среднее арифметическое всех метрик, собранных на шаге 5;
	\item Увеличивалось количество рассматриваемых параллельных соединений на 50;
	\item Шаги 1–6 повторялись до достижения 1500 параллельных соединений.
\end{enumerate}



\section{Результат исследования}

По результатам тестирования были построены графики: зависимость количества запросов в секунду от числа одновременных запросов к серверу(рисунок~\ref{img:rps_graph}) и зависимость скорости отдачи данных по каждому сетевому соединению и в совокупности(рисунок~~\ref{img:transfer_graph}).


\FloatBarrier
\imgw{1.0\textwidth}{rps_graph}{Зависимость количества запросов в секунду от числа одновременных запросов к серверу}
\FloatBarrier
\imgw{1.0\textwidth}{transfer_graph}{Зависимость скорости отдачи данных по каждому сетевому соединению и в совокупности}
\FloatBarrier

\section{Анализ результатов}
В ходе тестирования установлено, что критический порог производительности составляет 500 одновременных соединений, после которого наблюдается снижение эффективности работы сервера. При данной нагрузке максимальная скорость обработки данных достигает 170 МБайт/секунду.



\section{Вывод из исследовательской части}
В данном разделе было описано нагрузочное тестирование разработанного веб-сервера и определены максимальное количество обслуживаемых сетевых соединений и скорость отдачи данных по каждому сетевому соединению и совокупная.


\clearpage
