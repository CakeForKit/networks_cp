\chapter{Конструкторская часть}

В данном разделе представлено описание работы сервера.


\section{Описание алгоритма запуска сервера}

На рисунке представлено описание алгоритма запуска сервера~\ref{img:arch}
\FloatBarrier
\imgh{0.7\textheight}{arch}{Описание алгоритма запуска сервера}
\FloatBarrier

\section{Описание обработки запроса пользователя}

На рисунках~\ref{img:conn1}-\ref{img:conn2} представлено описание алгоритма обработки запроса пользователя.
\FloatBarrier
\imgh{0.8\textheight}{conn1}{Описание обработки запроса пользователя (часть 1)}
\FloatBarrier
\FloatBarrier
\imgh{0.8\textheight}{conn2}{Описание обработки запроса пользователя (часть 2)}
\FloatBarrier




%\begin{longtable}{|
%		>{\centering\arraybackslash}m{.33\textwidth - 2\tabcolsep}|
%		>{\centering\arraybackslash}m{.34\textwidth - 2\tabcolsep}|
%		>{\centering\arraybackslash}m{.33\textwidth - 2\tabcolsep}|
%	}
%	\caption{Таблица Коллекция (Collection)}\label{tbl:Collection} \\\hline
%	Поле & Описание & Тип и ограничение  \\\hline    
%	\endfirsthead
%	\caption*{Продолжение таблицы~\ref{tbl:Collection} } \\\hline
%	Поле & Описание & Тип и ограничение  \\\hline           
%	\endhead
%	\endfoot
%	idCollection & Идентификатор коллекции & Уникальное в таблице целое число \\\hline
%	nameCollection & Название коллекции & Строковое поле \\\hline
%\end{longtable}


\section*{Вывод}

В данном разделе были описаны начало работы сервера и алгоритм обработки запроса клиента.
