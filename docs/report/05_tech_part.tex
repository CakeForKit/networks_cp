\chapter{Технологическая часть}

В данном разделе представлены средства реализации и описано проведенное тестирование.

\section{Средства реализации}

Для написания веб-сервера использовались средства реализации:
\begin{itemize}
	\item Язык программирования: С~\cite{clang};
	\item Компилятор: gcc~\cite{gcc};
	\item Cистема сборки: Make~\cite{make};
\end{itemize}

Тестирование проводилось с использованием утилиты curl~\cite{curl}. 


\section{Тестирование}

Для функционального тестирования работы веб-сервера был написал скрипт, который отправляет запросы на сервера с помощью утилиты curl~\cite{curl}. Были рассмотрены классы эквивалентности, описанные в таблице~\ref{tbl:ftest}:

\begin{longtable}{|
		>{\centering\arraybackslash}m{.7\textwidth - 2\tabcolsep}|
		>{\centering\arraybackslash}m{.3\textwidth - 2\tabcolsep}|
	}
	\caption{Функциональные тесты}\label{tbl:ftest} \\\hline
	Описание теста & Ожидаемый код состояния HTTP \\\hline  
	\endfirsthead
	\caption*{Продолжение таблицы~\ref{tbl:ftest} } \\\hline
	Описание теста & Ожидаемый код состояния HTTP \\\hline  
	\endhead
	\endfoot
	HEAD запрос существующего файла размером до 128 Мбайт & 200 \\\hline
	GET запрос существующего файла размером до 128 Мбайт & 200 \\\hline
	Запрос несуществующего файла & 404 \\\hline
	Обращение к родительской директории (содержание двоеточия в пути) & 403 \\\hline
	Обращение к файлу доступ к которому запрещен & 403 \\\hline
	GET запрос существующего файла размером более 128 Мбайт & 403 \\\hline
	POST запрос & 405 \\\hline
	PUT запрос & 405 \\\hline
	DELETE запрос & 405 \\\hline
\end{longtable}

Все тесты пройдены успешно.


%\clearpage
\section*{Вывод}
Была представлены средства реализации и описано проведенное тестирование.

\clearpage
