\usepackage{cmap} 
\usepackage{tempora} 
\usepackage[T2A]{fontenc} 
\usepackage[utf8]{inputenc} 
\usepackage[english,russian]{babel} 
\usepackage{pdfpages} 
\usepackage{placeins} 
\usepackage[figure,table]{totalcount} 
\usepackage{lastpage} 
\usepackage{url}
\usepackage{xurl}
\urlstyle{same}

\usepackage{amssymb,amsfonts,amsmath,mathtext,float} % cite
\usepackage{letltxmacro} 
\usepackage{flafter} 

\usepackage[backend=biber, style=gost-numeric-inline]{biblatex}
\addbibresource{lib.bib}

\pdfcompresslevel=0
\pdfobjcompresslevel=0
%\pdfoptionpdfminorversion 7

\usepackage{geometry}
\geometry{left=30mm}
\geometry{right=15mm}
\geometry{top=20mm}
\geometry{bottom=20mm}

\setlength{\parindent}{1.25cm}

\usepackage{microtype}
\sloppy	

\usepackage{setspace}
\onehalfspacing 

\frenchspacing	
\usepackage{indentfirst} 


\makeatletter
\renewcommand\LARGE{\@setfontsize\LARGE{22pt}{20}}
\renewcommand\Large{\@setfontsize\Large{20pt}{20}}
\renewcommand\large{\@setfontsize\large{16pt}{20}}
\makeatother

\newcommand{\ssr}[1]{\begin{center}
		\large\bfseries{#1}
	\end{center} \addcontentsline{toc}{chapter}{#1}  }


\usepackage{titlesec}
\titleformat{\chapter}{\large\bfseries}{\thechapter}{14pt}{\large\bfseries}
\titleformat{name=\chapter,numberless}{}{}{0pt}{\large\bfseries\centering}	
\titleformat{\section}{\large\bfseries}{\thesection}{14pt}{\large\bfseries}
\titleformat{\subsection}{\large\bfseries}{\thesubsection}{14pt}{\large\bfseries}

\titlespacing{\chapter}{12.5mm}{-22pt}{10pt}
\titlespacing{\section}{12.5mm}{10pt}{10pt}
\titlespacing{\subsection}{12.5mm}{10pt}{10pt}

\usepackage{longtable} 
\usepackage{array}    
\usepackage{caption}
\captionsetup[longtable]{justification=raggedright, singlelinecheck=off}
\usepackage{multirow} 

\newcommand{\specialcell}[2][c]{
	\begin{tabular}[#1]{@{}c@{}}#2\end{tabular}}


\usepackage{enumerate} 


\usepackage{enumitem}
\def\labelitemi{---}
\setlist[itemize]{leftmargin=1.25cm, itemindent=0.65cm}
\setlist[enumerate]{leftmargin=1.25cm, itemindent=0.55cm}

\captionsetup{labelsep=endash}
\captionsetup[figure]{name={Рисунок}, justification=centering}


\usepackage{graphicx}
\newcommand{\imgScale}[3] {
	\begin{figure}[h!]
		\center{\includegraphics[scale=#1]{img/#2}} 
		\caption{#3}
		\label{img:#2}
	\end{figure}
}

\newcommand{\imgh}[3] {
	\begin{figure}[h!]
		\center{\includegraphics[height=#1]{img/#2}}
		\caption{#3}
		\label{img:#2}
	\end{figure}
}
\newcommand{\imgw}[3] {
	\begin{figure}[h!]
		\center{\includegraphics[width=#1]{img/#2}}
		\caption{#3}
		\label{img:#2}
	\end{figure}
}
\newcommand{\boximg}[3] {
	\begin{figure}[h]
		\center{\fbox{\includegraphics[height=#1]{img/#2}}}
		\caption{#3}
		\label{img:#2}
	\end{figure}
}



\usepackage{listings}
\usepackage{xcolor}
\lstset{
	basicstyle=\small\ttfamily,
	frame=single,
	captionpos=b,
	breaklines=true,
	tabsize=4
}

\newcommand{\dlr}{\mbox{\textdollar}}



\makeatletter
\renewcommand{\@biblabel}[1]{#1.}
\makeatother


\usepackage[unicode,pdftex]{hyperref}
\hypersetup{hidelinks}  

\usepackage{cleveref}

\makeatletter
\renewcommand*{\l@chapter}[2]{
	\ifnum \c@tocdepth>\m@ne
	\addpenalty{-\@highpenalty}
	\vskip 1em \@plus.2em
	\@dottedtocline{0}{0pt}{1.5em}{\bfseries #1}{\bfseries #2}
	\fi
}
\makeatother


\usepackage{listings}
\usepackage{xcolor}
\lstset{ %
	language=C++,                 % выбор языка для подсветки (здесь это С++)
	basicstyle=\small\sffamily, % размер и начертание шрифта для подсветки кода
	showspaces=false,            % показывать или нет пробелы специальными отступами
	showstringspaces=false,      % показывать или нет пробелы в строках
	showtabs=false,             % показывать или нет табуляцию в строках
	frame=single,              % рисовать рамку вокруг кода
	tabsize=2,                 % размер табуляции по умолчанию равен 2 пробелам
	captionpos=t,              % позиция заголовка вверху [t] или внизу [b] 
	breaklines=true,           % автоматически переносить строки (да\нет)
	breakatwhitespace=false, % переносить строки только если есть пробел
	escapeinside={\#*}{*)} ,  % если нужно добавить комментарии в коде
	keywordstyle=\color{purple},          % Стиль ключевых слов
	commentstyle=\color{dkgreen},       % Стиль комментариев
	stringstyle=\color{mauve}          % Стиль литералов
}